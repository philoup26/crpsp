\documentclass[11pt, A4]{article}

\usepackage{startup}
\usepackage{crla_titlepage}

\title{CRPSP}
\personaltitle{Cesium Radioactivity Perfect Secrecy Protocol \\
A privately usable system for secure communications}
% \classnb{<++>}
\subject{Personal Project}
% \group{<++>}
% \author{}
% \logo{}
\presentation{}
% \teacher{Nom de l'enseignant}
\programtype{\vspace{3cm}}
\institution{}
% \date{<++>}

\begin{document}
\maketitle
\tableofcontents
\newpage

\section{About}
%TODO: Write About
<++>

\section{Terminology}
%TODO: Write Terminology
<++>

\section{Usage}
<++>

\section{Process}
\subsection{Data Acquisition}
The first step in this protocol is acquiring purely random data. This is done
using HotBits: A non-lucrative organisation which gives random data generated
by a Cesium reactor. To do so, a micro-computer (Raspberry Pi) downloads
2048 bit packages of data up to a total of approximately 2 Gigabytes. The
packages are stored in an index folder which has a ``.crpsp'' file extension.
The 2048 bit packages are stored in a ``.xml'' file.

\subsubsection{Index Structure}
The index contains four main entities: \textbf{root}, \textbf{hierarchy},
\textbf{xml}, and \textbf{breadcrumbs}. \\

\textbf{Root:} This entity is a folder which holds breadcrumbs, the hierarchy,
as well as the log file. The folder itself must end with ``.crpsp''
for easier recognition. The purpose and structure of the log file will be
further discussed later. \\

\textbf{Hierarchy:} This section is the one in which all the xml packages are
organised. It consists of a set of 100 folders (labelled from 00-99)
each containing 100 folders (labelled in the same fashion) which each
contain 100 xml random data packages: the XML entity.\\  % TODO: Finish up

\textbf{XML:} This entity holds random data acquired from HotBits.
As it may very well be deduced,  it consists of an xml file for which the
structure is the following.  %TODO: Explain the xml structure

\textbf{Breadcrumbs:} This section is the one which reutilises the remains of
previously used xml packages. It serves as a middleground in which packages
with a length smaller than 2048 bits can be stored until they are recombined
into a new unique xml package which will be stored in the index. \\

\subsubsection{XML file Structure}
\begin{table}
    \caption{Xml File Structure}
    \label{xml_file_structure}
    \begin{center}
        \begin{tabular}{l | l |l l}
            \toprule
            \multicolumn{2}{c}{Tag} & Description \\
            \toprule
            hotbits & & & Root tag \\
                    & status & & \\
                    & request-information & <++> \\
                    & & server-version & <++> \\
                    & & generation-time & <++> \\
                    & & bytes-requested & <++> \\
                    & & bytes-returned & <++> \\
                    & & quota-request-remaining & <++> \\
                    & & quota-bytes-remaining & <++> \\
                    & & generator-type & <++> \\
                    & random-data &  & <++> \\
            \bottomrule
        \end{tabular}
    \end{center}
\end{table}

<++>

\end{document}
